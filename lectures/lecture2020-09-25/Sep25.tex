\documentclass{article}
\usepackage[utf8]{inputenc}
\usepackage{amssymb}
\usepackage{amsmath}
\title{September 25th Notes}
\author{Carter Blum}
\date{October 2020}

\newenvironment{parboxed}
    {\begin{center}
    \begin{tabular}{|p{0.9\textwidth}|}
    \hline\\
    }
    { 
    \\\\\hline
    \end{tabular} 
    \end{center}
    }
    
\newcommand{\mat}[1]{\begin{bmatrix} #1 \end{bmatrix}}


\begin{document}

\maketitle

\textbf{Big Idea: } Reach a goal while always satisfying constraints

\begin{parboxed}
\emph{Example} : Driving a car is constrained by the following:\\
\begin{center}
    $\cdot$ speed $\cdot$ steering angles $\cdot$ road limits (hopefully)
\end{center}
\end{parboxed}

In general, our goal will be to get to the origin \& stay there (we can always call our goal the origin by re-centering values. Staying at the origin entails ensuring stability there).\\

One strategy is to plan out our trajectory $N$ steps in the future, then execute the first control $u$ in that trajectory (then we repeat this process until we reach a desired state). 
This framework is broadly referred to as \textbf{\underline{Model Predictive Control}}.

Formally, for a given $x$, we solve 

$$\mathbb{P}_N = \min_u \sum_{i=0}^{N-1} l(x_i, u_i) + V_f(x_N)$$
\begin{equation*}
\begin{split}
        \textrm{s. t. } & x_0 = x\\
    (1) \quad & x_{i+1} = f(x_i, u_i)\\
    (2) \quad & (x_i, u_i) \in \mathbb{Z}\\
    (3) \quad & x_N \in \mathbb{X}_f
\end{split}
\end{equation*}

Where (1) represents the dynamics of the systems and (2) and (3) represent the additional constraints. $\mathbb{Z}$ here represents a set of acceptable values, not the set of integers.
The last constraint is usually for analytic purposes (e.g. stability guaranteed in that region, or its a region of attraction where we can easily apply LQR).
Here, $V_f$ is the ``Lyapunov Control Function".

\begin{parboxed}
\emph{Example} : Say $u:=kx$ with $|u| \leq 1$ as a constraint.
Then the above definition of $u$ will clearly violate the constraint for large values of $x$ - but we know that there is some region where $\|x\|$ is small so that the control is safe.
\end{parboxed}

Let $u^*(x) = \Big(u^*(0,x), ..., u^*(N-1,x)\Big)$ be an optimal input sequence for $\mathbb{P}_N(x)$. We will then set $u = u^*(0,x)=:\kappa(x)$.
You then plug in $x_{t+1} = f(x, u)$ as our new $x$ and repeat. 
This process can be really slow - it was originally meant for very slow-changing processes. 
Nowadays however, computers are extremely fast, so it has been sped up a lot.
Some sources of improvement are the following:

\begin{itemize}
    \item Specialized hardware 
    \item Only $x$ actually changes at each iteration
    \item $x$ typically changes quite minimally each iteration
    \item A lot of the past trajectory can be reused
    \item Common to stop planning early
\end{itemize}

One final thing to note: future constraints often depend on past decisions and values.

\begin{parboxed}
\emph{Example: } Consider an example where we're controlling a robotic arm. Let
$$x_t = \mat{\theta_t \\ \omega_t}$$
$$x_{t+1} = x_t + h\mat{\omega_t \\ \frac{u_t}{m}}$$
where $m$ is the mass. Let's say that we have constraints that $|\theta|\leq 5$ and $|u|\leq 1$, which correspond to the notions that the arm can only bend to certain angles and the robot can only exert so much force.
Then, for some initial state (like when $\omega$ is very large), it might be impossible to fulfil the constraints for all $t$.
It's also possible to encounter problems where the robot may need to start exerting force very early to slow down the angular momentum.
Failure to do so might result in later constraints being unsatisfied.
\end{parboxed}
\end{document}
